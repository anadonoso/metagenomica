\documentclass{article}
\usepackage[utf8]{inputenc}
\usepackage[spanish]{babel}

\title{Descripción de los objetivos concretos a alcanzar por la persona candidata}
\author{Donoso, Ana}

\begin{abstract}

\maketitle

Los rotavirus son uno de los principales agentes causantes de la gastroenteritis aguda (GEA) a nivel global. La enorme diversidad entre las cepas de rotavirus y su constante evolución supone un gran reto para los programas de vacunación del rotavirus. Ante la introducción de la vacuna en el calendario es importante tener más información sobre otros genotipos o especies de rotavirus que están circulando y que no estuviéramos detectando con los métodos convencionales. Por tanto, resulta necesario impulsar la vigilancia molecular mejorando la metodología para la caracterización de los rotavirus detectados y otros virus productores de GEAs (norovirus GIV y GIX, el virus de Aichi, etc) que están también causando clínica tal como ocurre en otros países de nuestro entorno (Rivadulla, et al. 2020; Tapparel C, et al., 2013).

Para ello, se han planteado dos abordajes: 1) el estudio en casos de GEA a los que no se ha llegado a ninguna filiación etiológica a pesar del empleo de paneles extensivos a los enteropatógenos más comunes; 2) el estudio en aguas residuales desarrollando nueva metodología de secuenciación masiva de detección para rotavirus en estas muestras ambientales. 

Estudiar la epidemiología molecular mediante secuenciación masiva tanto del rotavirus como otros virus productores de GEAs en muestras clínicas y ambientales nos permitiría conocer 1) los genotipos circulantes de rotavirus, 2) detectar y caracterizar a) nuevos tipos emergentes en nuestro entorno, b) la circulación de rotavirus derivados de vacuna productores de GEA, y c) los casos de fallos vacunales en menores hospitalizados tal como se recomienda desde el ECDC así como 3) describir los mecanismos evolutivos del rotavirus.

En nuestra Unidad del CNM se lleva a cabo la vigilancia de poliovirus y otros enterovirus en las aguas residuales de Madrid desde hace más de 20 años en colaboración con el Canal de Isabel II. Nuestro laboratorio realiza la técnica “gold standard” recomendada por la OMS que requiere del aislamiento del virus mediante cultivos celulares seguido de la tipificación molecular mediante RT-PCR y posterior secuenciación Sanger. La técnica mediante cultivo requiere de un tiempo de 2-3 semanas en promedio entre la recepción de la muestra y la secuenciación.   Además, esta metodología presenta otras limitaciones como puede ser la aparición de mutaciones del virus durante el crecimiento del cultivo celular, una menor sensibilidad y la falta de detección de variantes minoritarias.  

Por otro lado, la secuenciación masiva presenta numerosas ventajas respecto a las técnicas moleculares que son consideradas el “gold standard”. En concreto, el protocolo de secuenciación basado en la tecnología Nanopore permite la detección y secuenciación directa de estos virus con una mayor sensibilidad y un mayor rendimiento. Por tanto, esta técnica aceleraría significativamente el proceso de análisis y contribuiría a dar una respuesta más rápida en la identificación de virus relevantes en comparación con el método estándar de vigilancia de poliovirus establecido por la OMS. 

 \end{abstract}
